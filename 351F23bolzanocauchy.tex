\documentclass[11pt]{article}
\usepackage{amssymb, amsmath, amsthm}
\usepackage{simplemargins}
\usepackage{enumitem}
 \usepackage{mathtools}

\usepackage{graphicx}  % This one here.


\setlength{\parindent}{0in}
\setleftmargin{1.5in}
\setrightmargin{1.5in}
\settopmargin{1in}
\setbottommargin{1in}
\newcommand{\sk}{\vspace{10pt}}
\newcommand{\sm}{\vspace{2pt}}
\newcommand{\ds}{\displaystyle}
\newcommand{\real}{\mathbb{R}}
\newcommand{\que}{\mathbb{Q}}
\newcommand{\nat}{\mathbb{N}}
\newcommand{\zed}{\mathbb{Z}}

\newcommand{\e}{\varepsilon}



\newcommand{\limit[1]}{\underset{x\rightarrow #1}{\lim\; }}


\newcommand{\be}{\begin{enumerate}}
\newcommand{\ee}{\end{enumerate}}


\newtheorem{lem}{Lemma}
\newtheorem{prop}{Proposition}

\theoremstyle{definition}
\newtheorem{defn}{Definition}

\pagestyle{empty}
\begin{document}
\thispagestyle{empty}



\be

\item Discuss the following approaches to Theorem 5.10:

\be
\item  First, we prove that if a sequence converges to $L$, so do all subsequences. Since it is already true that for all $n>N, |a_n - L| < \e$. Then, it is known that for any $m_k>N$, this is also true, therefore $|a_{m_k} - L| < \e$ and $L$ is a limit for all subsequences. Going the other way, $(a_n)$ is a subsequence of $(a_n)$ using $m_k = k$. Then, if all subsequences of $(a_n)$ converge to $L$, and $(a_n)$ is a subsequence of $(a_n)$, $(a_n)$ converges to $L$. 

\item  <m>(\Rightarrow)</m> Suppose <m>(a_{n})</m> is a convergent sequence and <m>(a_{m_k})</m> is a subsequence of <m>(a_{n})</m>. By lemma 5.9 <m>m_{k} > k</m>, so <m>\forall k > N, m_{k} > N</m> so <m>|a_{m_k}- L| &#x3C; \e</m> holds.

$(\Leftarrow)$ Suppose subsequences $(a_{m_k})$ converge to $L$. This includes the subsequence $m_k = k$, such that $(a_{m_k}) = (a_k)$. Let $n = k$, so $a_n$ converges to $L$.
\ee

\item Discuss this approach to Exercise 5.11:  Suppose $(a_n)$ is a sequence and $(b_n)$ is the sequence of odd indexed terms of $(a_n)$ and $(c_n)$ is the sequence of even indexed terms of $(a_n)$. 

First direction: Suppose $(a_n)$ converges to $L$. Then by definition, for all epsilon, there exists a natural number $N$ where if $n > N$, $|a_n - L| < \e$. If $N$ is odd, then we see that that if $n > N, |b_n - L| < \e$ and if $N$ is even, if $n > N, |c_n - L| < \e$. Or we could just make $N + 1$ to account for the other case.

Second direction. Suppose $(b_n)$ and $(c_n)$ converge to $L$. Let $N = max(N_1, N_2)$ where $N_1$ is from $b_n$ converging and $N_2$ is converging, then we see for all $\e$, if $n > N, |a_n - L| < \e$. 


\item Discuss the following approaches to the Bolzano-Weierstrass Theorem for sets.  
\be

\item Given a bounded infinite subset $A$ of $\real$, then $A\subset I$ where $I_=[a,b]$ is a big enough closed interval. For a contradiction, assume that every element of $I$ is not an accumulation point of $A$.  Then for every $x\in I$, there is a number $\delta_x>0$ such that $N_{\delta_x}(x)\cap A\setminus{x}=\emptyset$.  Use this somehow to get a contradiction.

\item Given a bounded infinite subset $A$ of $\real$, consider two cases.  Case 1: There exists a nonempty subset $B$ of $A$ such that $B$ has no minimum.  We shall prove that the inf of $B$ is a limit point of $A$.    Case 2: Every nonempty subset of $A$ has a minimum.  Inductively define a set $B=\{ x_n\, |\, n\in \nat\}$ by $x_1$ is the minimum of $A$, $x_2$ is the minimum of $A-\{x_1\}$, $x_3$ is the minimum of $A-\{x_1,x_2\}$, etc.  We shall prove that $B$ has no maximum and that the sup of $B$ is a limit point of $A$.

\item Given a bounded infinite subset $A$ of $\real$, then $A\subset I_0$ where $I_0=[a_0,b_0]$ is a big enough closed interval.  Consider the sub-intervals $[a_0, c_0]$ and $[c_0,b_0]$ where $c_0$ is the midpoint of $I_0$.  If one of these contains infinitely many elements of $A$, call it $I_1$.  Repeat this process and use the Nested Interval Property and show that there must be an accumulation point somehow.

\ee


\ee



\end{document}




































% Oh wow are you still reading this?  Okay, well, keep scrolling down to find some crazy good hints.

































































































































































%  Keep going...























































































% More hints!















































































































































































































































































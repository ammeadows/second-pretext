\documentclass[12pt]{article}
\usepackage{amssymb, amsmath, amsthm}
\usepackage{simplemargins}
\usepackage{enumitem}
\setlength{\parindent}{0in}
\setleftmargin{1.5in}
\setrightmargin{1.5in}
\settopmargin{1in}
\setbottommargin{1in}
\newcommand{\sk}{\vspace{10pt}}
\newcommand{\sm}{\vspace{2pt}}
\newcommand{\ds}{\displaystyle}
\newcommand{\real}{\mathbb{R}}
\newcommand{\que}{\mathbb{Q}}
\newcommand{\nat}{\mathbb{N}}

\newcommand{\e}{\varepsilon}

\newcommand{\be}{\begin{enumerate}}
\newcommand{\ee}{\end{enumerate}}


\newtheorem{lem}{Lemma}
\newtheorem{prop}{Proposition}

\theoremstyle{definition}
\newtheorem{defn}{Definition}

\pagestyle{empty}
\begin{document}
\thispagestyle{empty}


\begin{center}
{\textsf{\Large MATH 351 Homework}}
\vspace{10pt}

{\textsf{\large Analysis \hfill Due at 2pm Monday 9/25}}
\end{center}

\sk

Read Chapter 2 of the book.  For this homework, using \LaTeX\ gives one point of extra credit for each problem.  







\sk

\renewcommand{\labelenumi}{\textbf{SF\theenumi.}}
\begin{enumerate}
\setcounter{enumi}{8}

\item Prove that if $A$ and $B$ are countably infinite sets, then so is $A\cup B$.

% It may be helpful to think of how the set of integers is countable.

\item Given any set $A$, consider the dodgeball set $D=\{X,O\}$.  Show that set $F$ consisting of all functions $f:A\rightarrow D$ has the same cardinality as the power set $\mathcal{P}(A)$ consisting of all subsets of $A$.

% You may want to associate a function f with the set of all elements x for which f(x)=X.

\end{enumerate}

\begin{defn}
Given a countable collection of sets $A_n$, $n\in \nat$, we define the intersection and union of these sets as 
\[
\bigcap_{n=1}^\infty A_n = \{x \, | \, \forall n\in \nat, x\in A_n\} \qquad \bigcup_{n=1}^\infty A_n = \{ x\, | \, \exists n\in \nat \hbox{ s.t.\ } x\in A_n\}
\]
\end{defn}

\begin{enumerate}

\setcounter{enumi}{13}    

\item Nested intervals
\be
\item Use one of the definition above to restate the Nested Interval Property.
% But which one????
\item Find with proof a sequence of nonempty nested intervals of the form $I_n=(a_n, b_n)$ such that $\bigcap_{n=1}^\infty I_n =\emptyset$.  
% OK now I feel like it was probably intersection.  Just have to find the intersection.  Wait, but it's infinite.  I don't know how to find the intersection of infinitely many sets.  Nobody ever defined this.  Ever.  Not even five minutes ago.
% I could try (1-1/n, 1+1/n).  Doesn't work though because every set contains the number 1.  (Yup, something about intersections and EVERY set).  But, it doesn't contain anything else besides 1 - why? Archimedean!
\item What does this say about the Nested Interval Property?
% With the NIP, the intersection was nonempty, but with this one the intersection was empty.  Surprising - is the NIP wrong then?  But we proved it!  Oh no math is broken!
\ee

\item Find with proof a sequence of (not necessarily nested) intervals of the form $I_n = [a_n, b_n]$ such that $\bigcup_{n=1}^\infty I_n = (0,1)$.

%Might be useful to make them nested the other way.  i.e. the n+1st interval containing the nth interval.

\item Consider the intervals $A_n = \{ x\in \real \, | \, x\ge n\}$.
\be
\item Are the sets $A_n$ nested?  Why or why not?  
% Sure, I can prove that the n+1st set is a subset of the nth.
\item Find with proof the intersection $\ds \bigcap_{n=1}^\infty A_n $.  
%  I bet 100 is in the intersection.  Every intersection always contains 100.  Wait - what about the 101st set?  Archimedes - get over here!
\item What does this say about the Nested Interval Property?
%  Ooh this feels just like the last time this question was asked.
\ee


\item Suppose $A$ is a nonempty bounded set of real numbers and $z$ is an upper bound for $A$.
\be
\item Prove or disprove: if $z\in A$, then $z=\sup(A)$.
% Why does it always say prove or disprove.  Why can't I prove AND disprove?  Feels like I could use the definition of least upper bound here.

\item Prove or disprove: if $z=\sup(A)$, then for every $\e>0$, there is some $x\in A$ such that $x>z-\e/2$.
% Finally, I've been wanting to say things about epsilon.
\ee


\end{enumerate}

\end{document}




































% Oh wow are you still reading this?  Okay, well, keep scrolling down to find some crazy good hints.



































































































































































% Is [0,1/n] interesting?

% What about [1/n, 1-1/n]?





















































































% More hints!






































































































































































































































































% Made you look.